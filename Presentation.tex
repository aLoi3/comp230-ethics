\documentclass{beamer}
\usepackage[english]{babel}
\setbeamertemplate{navigation symbols}{}
\setbeamertemplate{frametitle}[default][center]
\setbeamertemplate{bibliography item}{\insertbiblabel}
\usefonttheme[onlymath]{serif}

\usepackage{graphicx}
\graphicspath{ {./images/} }
\usepackage{amsmath}
\usepackage{wrapfig}

\title{Research Presentation}
\subtitle{COMP230}
\author{1702208}

\begin{document}

\begin{frame}
	\maketitle
\end{frame}

\begin{frame}{Background}
	Nowadays video games have become a very common way of spending time among adolescents. Playing games have both negative and positive effects. Positive effects are usually stress reduction\cite{russoniello2009effectiveness}, relaxation\cite{wack2009relationships}, emotional disturbances in children\cite{jones2014gaming}\cite{hull2009computer} and a lot more. But there are still some negative consequences. Besides the most common - addiction - there are aggression, antisocial behaviour and violence, that usually depends on the game itself.
\end{frame}

\begin{frame}{Massive Multiplayer Online Role-Playing Games}
	There are a lot of studies about MMORPGs, where the most common subject is WoW. It's a unique game to study for game ethics as it can bring both positive and negative results. The game is either a way to relax and reduce stress - usually for casual players - or ``risky addiction-like experiences"\cite{snodgrass2011magical}. 
\end{frame}

\begin{frame}{Violence}
	Violence in games usually depends on the game itself and how players will look at that. 
\end{frame}

\begin{frame}{Social or Antisocial?}
	
\end{frame}
\begin{frame}{References}
	\bibliographystyle{ieeetran}
	\bibliography{references}
\end{frame}
\end{document}